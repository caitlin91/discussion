% !TeX TXS-program:compile = txs:///pdflatex/[--shell-escape]
\documentclass[../../00.FullDoc/tex/ThesisSkeleton-draft2]{subfiles}


\title{Discussion}
\author{Caitlin Halfacre}
\date{\today}

\begin{document}

	\onlyinsubfile{
	\maketitle
	\pagebreak
	\tableofcontents
	\onehalfspacing
	\pagestyle{scrheadings}
				}
	
\section{Introduction} \label{sec:disc:Intro}
In this chapter I will discuss the results of chapters \ref{ch:FS},\ref{ch:TB}, and \ref{ch:GOAT}, bringing together the results of the analysis of all three variable to answer the research questions as outlined in chapter \ref{ch:Intro}.
I will then outline the limitations of the research, potential future work, and possible implications of the results.


\section{Research Questions} \label{sec:disc:RQs}
\begin{enumerate}
	\item In terms of pronunciation, do British north-eastern \quotesingle{RP} speakers behave regionally or non-regionally?
	Variables to be considered:
	\begin{itemize}
		\item \FS{} split
		\item \TB{} split
		\item \goat{} allophony
	\end{itemize}
	\item What insights do these differences give us into the nature of a non-regional sociolect comprising diffuse speech communities?
	\item If the results of \notinsubfile{\ref{RQ1}} \onlyinsubfile{1} vary depending on the variable in question, what phonological or social factors control this?
\end{enumerate}

\subsection{Research Question 1} \label{sec:disc:RQ1}
In terms of pronunciation, do British north-eastern \quotesingle{RP} speakers behave regionally or non-regionally?

\subsubsection{\scs{Foot}-\scs{Strut} Split}
In the analysis of the \FS{} split it was found that CoRP-SE and DECTE speakers behaved as expected. The CoRP-SE speakers show a split, which is mostly characterised by an F1 difference in, and some difference in F2. The DECTE speakers do not overall show evidence of a split.

The CoRP-NE speakers are variable with respect to the \FS{} split. Male speakers behave regionally, their \FS{} split patterns with the DECTE speakers. Female speakers do not behave regionally or non-regionally but instead have a different form of split. Competing inputs. Observing the CoRP-NE group shows a similar pattern to Labovian change from below the level of consciousness \todocontent{check the details of this, and adjust wording - is it reflective of, or a micro version?}, with female speakers tending towards a standard. However, there is no effect of style.

In conclusion, the male CoRP-NE speakers behave regionally, the female CoRP-NE speakers behave neither regionally nor non-regionally.

\subsubsection{\scs{Trap}-\scs{Bath} Split}
In the analysis of the \TB{} split, it was again found that the CoRP-SE and DECTE speakers behave as expected.
The CoRP-SE speakers show a \TB{} split, characterised by a difference in F2 and duration (and a small amount of F2 difference). \scs{Bath} is found to pattern with \palm{}. The DECTE speakers do not show any evidence of a split, again, as would be expected from speakers in the North of England. The \bath{} vowel in these speakers patterns with \trap{}. The DECTE speakers also show a \palm{} vowel that is further forward than in the CoRP-SE speakers. The CoRP-NE speakres can be described as behaving broadly regionally, with no clear \TB{} split. However, they do oeverall show a broader range of F2 values in the \bath{} words and also a longer duration. It is possible that some individual words \bath{} words are being produced with a \palm{}-like vowel, causing the variation seen in F2. There is also evidence of pre-fricative lengthening occurring in the CoRP-NE speakers, implying that a split could be starting in this speech community.

In conclusion, the CoRP-NE speakers can be described as behaving broadly regionally but with some indications non-regional behaviour or a change in progress towards non-regional behaviour.

\subsubsection{\scs{Goat} allophony}
In the analysis of the \goat{} vowel, CoRP-SE speakers were found to have a diphthong in all contexts, and a \GG{} split found in F2. With respect to the morphological conditioning of the \GG{} split the CoRP-SE speakers seem to sit at a mixture of stage 1 and stage 2 of the life cycle of phonological processes (\citealt{BermudezOtero2015a}, table \ref{tbl:lifecycle}). The DECTE speakers show a monophthongal \goat{} vowel and no evidence of a \GG{} split in any morphological context.

CoRP-NE speakers show non-regional behaviour with respect to the monosyllabic \goat{} vowel and the \GG{} split, but have different morphological conditioning, showing stage 3 of the life cycle, implying either a different acquisition of the morphological conditioning of the split or a more advanced progression of the change, either of which are non-regional behaviour.

\subsection{Research Question 2} \label{sec:disc:RQ2}
Research question 2 brings together the detailed results and leads us to consider the nature of RP, as discussed in sociolinguistic literature and as understood from the speakers analysed in this project. 
There are two particularly prevelant theories discussed in the literature regarding RP. First, as indicated by the diagram below, that regional variation decreases up the socioeconomic spectrum \todoreference{reference Wells and others}. In comparing speakers privately educated in the North East to those privately educated and finding regional variation in the CoRP-NE speakers,two possible conclusions are available:
\begin{enumerate}
	\item that a \quotesingle{non-regional} accent can only be found in the North East by going further up the socio-economic spectrum than is needed in the South East
	\newline OR
	\item that a regional accent does not exist, instead there is high social class/private education accent found in each area, which shows less regional variation but not none
\end{enumerate}
\todocontent{triangle img}

Secondly, that change in RP happens from innovations in working class South East accents moving up the socio-economic spectrum and then spreading across the country. This hypothesis is not supported by the results of this project. The \FS{} and \TB{} splits are not recent innovations, the changes have been established long enough in the South East and RP to expect them to exist in other areas. However, the \FS{} split is not found at all in male CoRP-NE speakers and is found in a different form in the female speakers. The \TB{} split is not found in vowel quality of the CoRP-NE speakers, the length distinction found is more likely to be a phonetic effect (which may lead to a change locally) than the beginning of the change spreading from the South East. The \GG{} split is more recent (\ref{ch:LitReview}), and the morphological conditioning can indicate the progress through a change via the life cycle of phonological processes. The CoRP-NE speakers show a stage of the split that is either a simplified version of the southern split, or further ahead of of the change. Either of these possibilities is not the innovation process suggested by Trudgill.

\todocontentinline{\goal{} variation?}

\subsection{Research Question 3} \label{sec:disc:RQ3}

\FS{} - speaker sex, male speakers regional, female speakers non-regional, no controlling phonological factors
\TB{} - speakers do not vary but some evidence for change (length)
\GG{} - behave non-regionally in simple split, complex morphology causes variation in \holy{} vs \holey{} 

There is not a clear pattern of overall regional or non-regional behaviour in the CoRP-NE speakers, instead we see broadly regional behaviour in the \TB{} split (with some caveats, as described above), regional behaviour in the \FS{} split in the male speakers. We see non-regional behaviour in the simple \GG{} split. However, we also find patterns that don't fit into either of those categories. The female speakers show a \strutt{} vowel that does not appear in either the CoRP-SE or DECTE speakers and the pattern of the morphological conditioning of the \GG{} split in disyllabic contexts seen in the CoRP-NE speakers is not the same as in either CoRP-SE or DECTE. However, it is also not clear that there is a particular social or linguistic context that conditions regional or non-regional behaviour. The data collected here cannot answer what pattern is governing this variation but some suggestions are made in the further work section below ().


%
%\subsubsection{\scs{Foot-Strut} split}
%The \FS{} split is consistent in the CoRP-NE and DECTE speakers; in the CoRP-NE speakers is not conditioned by any phonological factors but clearly affected by speaker sex. Male speakers do not show a split (\foot{} and \strutt{} words have the same vowel); female speakers do show a split but their \strutt{} vowel (and hence the size of the split) is not the same as that found in the CoRP-SE speakers.
%
%\subsubsection{\scs{Trap-Bath} split}
%
%\subsubsection{\scs{Goat} allophony}
%
%
%\TB{} - there is variation within CoRP-NE speakers. It's possible that this is word specific or down to word-internal factors. e.g. whether the fricative that follows is in the coda position or the onset of the next syllable - class vs castle - castle more likely to be in \palm{} position. But \bath{} is a small lexical set and there isn't enough data to defend work specific or syllable structure effects for certain.
%\todocontent{include model?}
%\palm{} difference
%length difference


\section{notes}
The CoRP-NE speakers are highly likely to have competing inputs - e.g. high mobility of the middle classes means that parents and caregivers are more likely to be from outside the North East \todocontent{list from participant forms}, school teachers and peers at school will also be a mixture.
Mixture of friends from area and school. 
Mobility within the city, e.g. primary and secondary school in different areas to each other and home, is likely to reduce the existence of very localised accent groups. 

\section{Other influences/issues}
Change in standard language ideology?
Fewer children moving across the country for education

\subsection{Limitations}
There were multiple limitations in the design and analysis of this study; chapter \ref{ch:Methodology} explained the rationale for speaker selection but there were still issues surrounding the speaker selection and categorisation. 
When defining South East versus North East there is the question of whether to use the area the person lived with their family or where they were educated. I chose to use the region of education, and only one speaker (008\_MO) had a discrepancy between the two. This issue leads into the question of boarding. Four of the CoRP-SE speakers boarded, two of the CoRP-NE speakers boarded, one in the North East and one in the South East before moving to a day school in the North East. There are also far few boarding schools in the North East overall. It is unquestionable that boarding had a significant influence on the development of RP but the mobility of these speakers (through both education and their adult lives) is not addressed when the term \quotesingle{non-regional} is used. Further investigation into the impact of mobility is needed but due to constraints of time and sample size was not possible in this study.

\section{Future Work} \label{disc:future}
This section will suggest a set of potential future directions that this research could take, particularly focussing on ways to move further towards answers to research questions \ref{RQ2} and \ref{RQ3}.

The research covered in this thesis was a high level sociolinguistic study and did not consider detailed social network influences. Further work could include mapping of social networks \citep{Milroy1992} in both childhood and present (from the participants presented here and future recordings) to further understand the nature of the speech community at a local level and discover the influences on individual's accents. Alongside this, collecting data from teen speakers, particularly from the CoRP-NE speaker group, while they are at the private schools will help to understand if the variation seen is caregiver or school effect, or change through the lifespan.

The variation in regional versus non-regional behaviour does not follow any clear pattern based on the data that has been analysed. However there are various avenues for understanding the variation. The first is social saliency and attitudes. The \TB{} split is very salient and generally marks northern identity, the \FS{} is less socially salient but is different marked, \citet{Wells1982b} describes not having the split as \quotemarks{vulgar}. Comparing these two data sets it could be suggested that the more marking of a regional identity a variant is, the more likely speakers are to retain it, even when they don't have other regional variants. Anecdotally a front \goat{} diphthong is marked in the south of England but there is very little discussion of the social saliency of the original vowel or the split across the country. More work on understanding the saliency and social meaning of these three variables needs to be undertaken to develop this possibility further. One way that this could be done could be using matched guise tests that manipulate the \bath{}, \strutt{}, and \goat{} vowels and test people's perceptions and associations with the variants.

This project did not consider individual behaviour, but a related possibility with respect to the lack of clear pattern that predicts regional behaviour is that individuals are working on a hierarchy of variables. For example, if we rank the regionality of the variables as diphthongal \goat{} (and \GG{} split) > \FS{} split > \TB{} split, any speaker that has a \TB{} split will also have a \FS{} split and diphthongal \goat{}, a speaker could not have a \TB{} split but have a \FS{} split and monophthongal \goat{}, or a speaker could not have a \TB{} split or a \FS{} split but still have a diphthongal \goat{}. Further research would work on analysis of each individual and their behaviour with respect to each variable. If an order can be deduced, the next question would be what controls that order, results from the social saliency analysis mentioned above could give insight into this.




\section{Implications}
discrimination

\onlyinsubfile{
	\listoftodos
	\pagebreak
	\bibliography{../../References/methodology,
		../../References/rRP,
		../../References/goatAllophony,
		../../References/tynesideEnglish,
		../../References/trapBath,
		../../References/footStrut
	}
}
	


\end{document}